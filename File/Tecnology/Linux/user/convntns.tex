% Linux Installation, Setup, and Getting Started    -*- TeX -*-
% Documentation conventions (convntns.tex)
% Copyright 1992, 1993 Michael K. Johnson, to be distributed freely.
% Use as directed ;-)
% modified by Larry Greenfield, 1994

These are some of the typographical conventions used in this book.

\begin{dispitems}
\item [{\bf Bold}] Used to mark {\bf new concepts}, {\bf WARNINGS},
and {\bf keywords} in a language.

\item [{\em italics}] Used for {\em emphasis} in text. 

\item [{\sl slanted}] Used to mark {\bf meta-variables} in the text,
especially in representations of the command line.  For example,
``{\tt ls -l} {\sl foo}''
where {\sl foo\/} would ``stand for'' a filename, such as {\tt /bin/cp}.

\item [{\tt Typewriter}] Used to represent screen interaction.

Also used for code examples, whether it is ``C'' code, a shell script,
or something else, and to display general files, such as configuration
files.  When necessary for clarity's sake, these examples or figures
will be enclosed in thin boxes.

\item [\key{Key}] Represents a key to press.  You will often see it
in this form: ``Press \ret\ to continue.''
% note that this is in a san-serif font...

\item [\hfill$\Diamond$] A diamond in the margin, like a
black diamond on a ski hill, marks ``danger'' or ``caution.''  Read
paragraphs marked this way carefully.

\item [\hfill\usebox{\xlogo}] This X in the margin indicates special
  instructions for users of the X Window System.\index{X Window System}

\item [\hfill\usebox{\caution}] This indicates a paragraph that
  contains special information that should be read carefully.

\end{dispitems}
% Local Variables: 
% mode: latex
% TeX-master: "guide"
% End: 
