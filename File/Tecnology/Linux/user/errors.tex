\chapter{Errors, Mistakes, Bugs, and Other Unpleasantries}\label{error-chapter}

\begin{quote}
\unix\ was never designed to keep people from doing stupid things,
because that policy would also keep them from doing clever things.\\
\raggedleft Doug Gwyn
\end{quote}

\section{Avoiding Errors}

Many users report frustration with the \unix\ operating system at one
time or another, frequently because of their own doing.  A feature of
the \unix\ operating system that many users' love when they're working
well and hate after a late-night session is how very few commands ask
for confirmation.  When a user is awake and functioning, they rarely
think about this, and it is an assest since it let's them work
smoother.

However, there are some disadvantages.  {\tt rm} and {\tt mv} never
ask for confirmation and this frequently leads to problems.  Thus,
let's go through a small list that might help you avoid total
disaster:

\begin{itemize}
\item Keep backups!  This applies especially to the one user
  system---all system adminstrators should make regular backups of
  their system!  Once a week is good enough to salvage many files.
  See the \ldpsa\ for more information.
\item Individual user's should keep there own backups, if possible.
  If you use more than one system regularly, try to keep updated
  copies of all your files on each of the systems.  If you have access
  to a floppy drive, you might want to make backups onto floppies of
  your critical material.  At worst, keep additional copies of your
  most important material lying around your account {\em in a seperate
    directory\/}!
\item Think about commands, especially ``destructive'' ones like {\tt
    mv}, {\tt rm}, and {\tt cp} before you act.  You also have to be
  careful with redirection ({\tt >})---it'll overwrite your files when
  you aren't paying attention. Even the most harmless of commands can
  become sinister:
\begin{screen}\begin{verbatim}
/home/larry/report# cp report-1992 report-1993 backups
\end{verbatim}\end{screen}
can easily become disaster:
\begin{screen}\begin{verbatim}
/home/larry/report# cp report-1992 report-1993
\end{verbatim}\end{screen}
\item The author also recommends, from his personal experience, not to
  do file maintanence late at night.  Does you directory structure
  look a little messy at 1:32am?  Let it stay---a little mess never
  hurt a computer.
\item Keep track of your present directory.  Sometimes, the prompt
  you're using doesn't display what directory you are working in, and
  danger strikes.  It is a sad thing to read a post on {\tt
    comp.unix.admin}\footnote{A international discussion group on
    Usenet, which talks about administring \unix\ computers.} about a
  {\tt root} user who was in {\tt /} instead of {\tt /tmp}!  For
  example:
  \begin{screen}\begin{verbatim}
mousehouse> pwd
/etc
mousehouse> ls /tmp
passwd
mousehouse> rm passwd
\end{verbatim}\end{screen}

The above series of commands would make the user very unhappy, seeing
how they have just removed the password file for their system. Without
it, people can't login!
\end{itemize}

\section{What to do When Something Goes Wrong}



\section{Not Your Fault}

Unfortunately for the programmers of the world, not all problems are
caused by user-error. \unix\ and \linux\ are complicated systems, and
all known versions have bugs.  Sometimes these bugs are hard to find
and only appear under certain circumstances.  

First of all, what is a bug?  An example of a bug is if you ask the
computer to compute ``5+3'' and it tells you ``7''. Although that's a
trivial example of what can go wrong, most bugs in computer programs
involve arithmetic in some extremely strange way. 

\subsection{When Is There a Bug}

If the computer gives a wrong answer (verify that the answer is
wrong!) or crashes, it is a bug. If any one program crashes or gives
an operating system error message, it is a bug.

If a command never finishes running can be a bug, but you must make
sure that you didn't tell it to take a long time doing whatever you
wanted it to do.  Ask for assistance if you didn't know what the
command did.

Some messages will alert you of bugs.  Some messages are not bugs.
Check Section~\ref{kernel-messages} and any other documentation to
make sure they aren't normal informational messages.  For instance,
messages like ``disk full'' or ``lp0 on fire'' aren't software
problems, but something wrong with your hardware---not enough disk
space, or a bad printer.

If you can't find anything about a program, it is a bug in the
documentation, and you should contact the author of that program and
offer to write it yourself. If something is incorrect in existing
documentation\footnote{Especially this one!}, it is a bug with that
manual. If something appears incomplete or unclear in the manual, that
is a bug.

If you can't beat {\tt gnuchess}\ttindex{gnuchess} at chess, it is a
flaw with your chess algorithm, but not necessarily a bug with your
brain.

\subsection{Reporting a Bug}\label{error-reporting}

After you are sure you found a bug, it is important to make sure that
your information gets to the right place.  Try to find what program is
causing the bug---if you can't find it, perhaps you could ask for help
in {\tt comp.os.linux.help} or {\tt comp.unix.misc}. Once you find the
program, try to read the manual page to see who wrote it.

The preferred method of sending bug reports in the \linux\ world is
via electronic mail. If you don't have access to electronic mail, you
might want to contact whoever you got \linux\ from---eventually,
you're bound to encounter someone who either has electronic mail, or
sells \linux\ commercially and therefore wants to remove as many bugs
as possible.  Remember, though, that no one is under any obligation to
fix any bugs unless you have a contract!

When you send a bug report in, include all the information you can
think of.  This includes:
\begin{itemize}
\item A description of what you think is incorrect.  For instance,
  ``I get 5 when I compute 2+2'' or ``It says {\tt segmentation
    violation -- core dumped}.'' It is important to say exactly what
  is happening so the maintainer can fix {\em your\/} bug!
\item Include any relevant environment variables.
\item The version of your kernel (see the file {\tt /proc/version})
  and your system libraries (see the directory {\tt /lib}---if you
  can't decipher it, send a listing of {\tt /lib}).
\item How you ran the program in question, or, if it was a kernel bug,
  what you were doing at the time.
\item {\bf All} peripheral information.  For instance, the command
  {\tt w}\ttindex{w} may not be displaying the current process for
  certain users. Don't just say, ``{\tt w} doesn't work when for a
  certain user''.  The bug could occur because the user's name is
  eight characters long, or when he is logging in over the network.
  Instead say, ``{\tt w} doesn't display the current process for user
  {\tt greenfie} when he logs in over the network.''
\item And remember, be polite.  Most people work on free software for
  the fun of it, and because they have big hearts.  Don't ruin it for
  them---the \linux\ community has already disillusioned too many
  developers, and it's still early in \linux's life!
\end{itemize}

% Local Variables: 
% mode: latex
% TeX-master: "guide"
% End: 
