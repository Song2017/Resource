\chapter{Talking to Others}\label{network-chapter}

% should we cover Mosaic?

\begin{quote}
``One basic notion underlying Usenet is that it is a cooperative.''

Having been on Usenet for going on ten years, I disagree with
this. The basic notion underlying Usenet is the flame.\\
\raggedleft Chuq Von Rospach
\end{quote}

Modern \unix\ operating systems are very good at talking to other
computers, or networking. Two different \unix\ computers can
exchange information in many, many different ways.  This chapter is
going to try to talk about how you can take advantage of that strong
network ability.

We'll try to cover electronic mail, Usenet news, and several basic
\unix\ utilities used for communication.

\section{Electronic Mail}

One of the most popular standard features of \unix\ is electronic
mail. With it, you are spared the usual hassle of finding an envelope,
a piece of paper, a pen, a stamp, and the postal service, and,
instead, given the hassle of negotiating with the computer.

\subsection{Sending Mail}

All you need to do is type {\tt mail {\sl username}}\impttindex{mail}
and type your message.

For instance, suppose I wanted to send mail to a user named {\tt sam}:

\begin{screen}\begin{verbatim}
/home/larry# mail sam
Subject: The user documentation
Just testing out the mail system.
EOT
/home/larry#
\end{verbatim}\end{screen}

The {\tt mail}\ttindex{mail} program is very simple. Like {\tt cat},
it accepts input from standard input, one line at a time, until it
gets the end-of-text character on a line by itself: \eof. So, to send
my message off I had to hit return and then \eof.

{\tt mail} is the quickest way to send mail, and is quite useful when
used with pipes and redirection. For instance, if I wanted to mail
the file {\tt report1} to ``Sam'', I could {\tt mail sam < report1},
or I could have even run ``{\tt sort report1 | mail sam}''.

However, the downside of using {\tt mail} to send mail means a very
crude editor. You can't change a line once you've hit return! So, I
recommend you send mail (when not using a pipe or redirection) is with
Emacs's\index{Emacs} mail mode.  It's covered in
Section~\ref{emacs-mail-mode}.

\subsection{Reading Mail}

\begin{command}
  {\tt mail}\impttindex{mail} [{\sl user}]
\end{command}

The {\tt mail} program offers a clumsy way of reading mail. If you
type {\tt mail} without any parameters, you'll see the following:

\begin{screen}\begin{verbatim}
/home/larry# mail
No mail for larry
/home/larry#
\end{verbatim}\end{screen}

I'm going to send myself some mail so I can play around with the
mailreader:

\begin{screen}\begin{verbatim}
/home/larry# mail larry
Subject: Frogs!
and toads!
EOT
/home/larry# echo "snakes" | mail larry
/home/larry# mail
Mail version 5.5 6/1/90.  Type ? for help.
"/usr/spool/mail/larry": 2 messages 2 new
>N  1 larry                 Tue Aug 30 18:11  10/211   "Frogs!"
 N  2 larry                 Tue Aug 30 18:12   9/191  
& 
\end{verbatim}\end{screen}

The prompt inside the mail program is an ampersand (``{\tt \&}''). It
allows a couple of simple commands, and will give a short help screen
if you type {\tt ?} and then \ret.

The basic commands for {\tt mail} are:

\begin{dispitems}
\item [{\tt t} {\sl message-list}] Show (or {\bf t}ype) the messages
  on the screen.
\item [{\tt d} {\sl message-list}] Delete the messages.
\item [{\tt s} {\sl message-list} {\sl file}] Save the messages into
  {\sl file}.
\item [{\tt r} {\sl message-list}] Reply to the messages---that is,
  start composing a new message to whoever sent you the listed
  messages.
\item [{\tt q}] Quit and save any messages you didn't delete into a
  file called {\tt mbox} in your home directory.
\end{dispitems}

What's a {\sl message-list}? It consists of a list of integers
seperated by spaces, or even a range, such as {\tt 2-4} (which is
identical to ``{\tt 2 3 4}''). You can also enter the username of the
sender, so the command {\tt t sam} would {\bf t}ype all the mail from
Sam. If a message list is omitted, it is assumed to be the last
message displayed (or typed).

There are several problems with the {\tt mail} program's reading
facilities. First of all, if a message is longer than your screen, the
mail program doesn't stop! You'll have to save it and use {\tt more}
on it later. Second of all, it doesn't have a very good interface for
old mail---if you wanted to save mail and read it later.

Emacs also has a facility for reading mail, called {\tt rmail}, but it
is not covered in this book. Additionally, most Linux systems have
several other mailreaders available, such as {\tt elm}\ttindex{elm} or
{\tt pine}\ttindex{pine}.

% what about the .forward file!!! ****

\section{More than Enough News}

% nn?

\section{Searching for People}

\subsection{The {\tt finger} command}

The {\tt finger} command allows you to get information on other users
on your system and across the world.  Undoubtably the {\tt finger}
command was named based on the AT\&T\index{AT\&T} advertisements
exhorting people to ``reach out and touch someone''.  Since \unix\ has
its roots in AT\&T, this was probably amusing to the author.

\begin{command}
  {\tt finger}\impttindex{finger} [-slpm] [{\sl user}][{\sl @machine}]
\end{command}

The optional parameters to {\tt finger} may be a little
confusing. Actually, it isn't that bad. You can ask for information on
a local user (``sam''), information on another machine
(``@lionsden''), information on a remote user (``sam@lionsden''), and
just information on the local machine (nothing).

Another nice feature is, if you ask for information about a user and
there isn't an account name that is exactly what you asked for, it
will try and match the real name with what you specified. That would
mean that if I ran {\tt finger Greenfield}, I would be told that the
account {\tt sam} exists for Sam Greenfield.

\begin{screen}\begin{verbatim}
/home/larry# finger sam
Login: sam                              Name: Sam Greenfield
Directory: /home/sam                    Shell: /bin/tcsh
Last login Sun Dec 25 14:47 (EST) on tty2
No Plan.
/home/larry# finger greenfie@gauss.rutgers.edu
[gauss.rutgers.edu]
Login name: greenfie                    In real life: Greenfie
Directory: /gauss/u1/greenfie           Shell: /bin/tcsh
On since Dec 25 15:19:41 on ttyp0 from tiptop-slip-6439
13 minutes Idle Time
No unread mail
Project: You must be joking!
No Plan.
/home/larry# finger
Login    Name                 Tty  Idle  Login Time   Office     Office Phone
larry    Larry Greenfield      1   3:51  Dec 25 12:50
larry    Larry Greenfield      p0        Dec 25 12:51
/home/larry# 
\end{verbatim}\end{screen}

The {\tt -s} option tells {\tt finger} to always display the short
form (what you normally get when you finger a machine), and the {\tt
  -l} option tells it to always use the long form, even when you
finger a machine. The {\tt -p} option tells {\tt finger} that you
don't want to see {\tt .forward}, {\tt .plan}, or {\tt .project}
files, and {\tt -m} tells {\tt finger} that, if you asked for
information about a user, only give information about an account
name---don't try to match the name with a real name.

\subsection{Plans and Projects}

Now, what's a {\tt .plan}\ttindex{.plan} and a {\tt
  .project}\ttindex{.project}, anyway? They're files stored in a
user's home directory that are displayed whenever they're fingered.
You can create your own {\tt .plan} and {\tt .project} files---the
only restriction is that only the first line of a {\tt .project} file
is displayed.

Also, everybody must have execute privileges in your home directory
({\tt chmod a+x \verb+~+/}) and everybody has to be able to read the
{\tt .plan} and {\tt .project} files ({\tt chmod a+r \verb+~+/.plan
  \verb+~+/.project}).

\section{Using Systems by Remote}

% telnet, rlogin

\begin{command}
  {\tt telnet}\impttindex{telnet} {\sl remote-system}
\end{command}

The principal way of using a remote \unix\ system is through {\tt
  telnet}.  {\tt telnet} is usually a fairly simple program to use:

\begin{screen}\begin{verbatim}
/home/larry# telnet lionsden
Trying 128.2.36.41...
Connected to lionsden
Escape character is '^]'.

lionsden login: 
\end{verbatim}\end{screen}
  
As you can see, after I issue a {\tt telnet} command, I'm presented
with a login prompt for the remote system.  I can enter any username
(as long as I know the password!) and then use that remote system
almost the same as if I was sitting there.

The normal way of exiting {\tt telnet} is to {\tt logout} on the
remote system, but another way is to type the escape character, which
(as in the example above) is usually \key{Ctrl-]}.  This presents me
with a new prompt titled {\tt telnet>}.  I can now type {\tt quit} and
\ret and the connection to the other system will be closed and {\tt
  telnet} will exit.  (If you change your mind, simply hit return and
you'll be returned to the remote system.)

\xwarn If you're using X, let's create a new {\tt xterm} for the other
system we're travelling to. Use the command ``{\tt xterm -title
  "lionsden" -e telnet lionsden \&}''\ttindex{xterm}. This will create
a new {\tt xterm} window that's automatically running {\tt telnet}.
(If you do something like that often, you might want to create an
alias or shell script for it.)

\section{Exchanging Files}

% ftp

\begin{command}
  {\tt ftp}\impttindex{ftp} {\sl remote-system}
\end{command}

The normal way of sending files between \unix\ systems is {\tt ftp},
for the {\bf file transfer protocol}.  After running the {\tt ftp}
command, you'll be asked to login to the remote system, much like {\tt
  telnet}.  After doing so, you'll get a special prompt: an {\tt ftp}
prompt.

The {\tt cd} command works as normal, but on the remote system: it
changes your directory on the \emph{other} system.  Likewise, the {\tt
  ls} command will list your files on the remote system.

The two most important commands are {\tt get} and {\tt put}.  {\tt
  get} will transfer a file from the remote system locally, and {\tt
  put} will take a file on the local system and put in on the remote
one.  Both commands work on the directory in which you started {\tt
  ftp} locally and your current directory (which you could have
  changed through {\tt cd}) remotely.

One common problem with {\tt ftp} is the distinction between text and
binary files.  {\tt ftp} is a very old protocol, and there use to be
advantages to assuming that files being transferred are text files.
Some versions of {\tt ftp} default to this behavior, which means any
programs that get sent or received will get corrupted.  For safety,
use the {\tt binary} command before using {\tt get} or {\tt put}.

To exit {\tt ftp} use the {\tt bye} command.

\section{Travelling the Web}

World Wide Web, or WWW, is a popular use of the Internet.  It consists
of {\bf pages}, each associated with its own URL---{\bf uniform
  resource locator}.  URLs are the funny sequence of in the form {\tt
  http://www.rutgers.edu/}.  Pages are generally written in HTML ({\bf
  hypertext markup language}).
\index{URL}\glossary{URL}\glossary{uniform resource locator}
\glossary{HTML}\glossary{hypertext markup language}

HTML allows the writer of a document to link certain words or phrases
(or pictures) to other documents anywhere else in the Web.  When a
user is reading one document, she can quickly move to another by
clicking on a key word or a button and been presented with another
document---possibly from thousands of miles away.

\begin{command}
  {\tt netscape} [{\sl url}]
\end{command}

\xwarn The most popular web browser on \linux\ is {\tt
  netscape}\ttindex{netscape}, which is a commercial browser sold (and
given away) by Netscape Communications Corporation.  {\tt netscape}
only runs under X.

{\tt netscape} tries to be as easy to use as possible and uses the
Motif widget set to display a very Microsoft Windows-like appearance.
The basic strategy for using {\tt netscape} is that underlined blue
words are links, as are many pictures.  (You can tell which pictures
are links by clicking on them.)  By clicking on these words with your
left mouse button, you'll be presented with a new page.

\linux\ supports many other browsers, including the original web
browser {\tt lynx}\ttindex{lynx}.  {\tt lynx} is a text browser---it
won't display any of the pictures that the Web is currently associated
with---but it will work without X.

\begin{command}
  {\tt lynx}\impttindex{lynx} [{\sl url}]
\end{command}

It's somewhat harder to learn how to use {\tt lynx}, but generally
playing with the arrow keys will let you get the hand of it.  The up
and down arrow keys move between links on a given page, which the
right arrow key follows the current (highlighted) link.  The left
arrow key will reload the previous page.  To quit {\tt lynx}, type
\key{q}.  {\tt lynx} has many other key commands---consult the manpage
for more.


% Local Variables: 
% mode: latex
% TeX-master: "guide"
% End: 
