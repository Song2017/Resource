\chapter{What's \unix, anyway?}\label{unixinfo}

\begin{fortune}
Ken Thompson\index{Thompson, Ken} has an automobile which he helped
design.  Unlike most automobiles, it has neither speedometer, nor gas
gage, nor any of the numerous idiot lights which plague the modern
driver.  Rather, if the driver makes any mistake, a giant ``?'' lights
up in the center of the dashboard.  ``The experienced driver,'' he
says, ``will usually know what's wrong.''
\end{fortune}

\section{\unix\ History}
In 1965, Bell Telephone Laboratories (Bell Labs, a division of
AT\&T\index{AT\&T})\glossary{Bell Labs} was working with General
Electric\index{General Electric} and Project MAC of
MIT\index{Massachusetts Institute of Technology} to write an operating
system called Multics.\index{Multics} To make a long story slightly
shorter, Bell Labs decided the project wasn't going anywhere and broke
out of the group.  This left Bell Labs without a good operating
system.

Ken Thompson\index{Thompson, Ken} and Dennis Ritchie\index{Ritchie,
  Dennis} decided to sketch out an operating system that would meet
Bell Labs' needs.  When Thompson needed a development environment
(1970) to run on a PDP-7, he implemented their ideas.  As a pun on
Multics, Brian Kernighan\index{Kernighan, Brian}, another Bell Labs
researcher, gave the system the name \unix.

Later, Dennis Ritchie\index{Ritchie, Dennis} invented the ``C''
programming language. In 1973, \unix\ was rewritten in C instead of
the original assembly language.\footnote{``Assembly language'' is a
  very basic computer language that is tied to a particular type of
  computer.  It is usually considered a challenge to program in.} In
1977, \unix\ was moved to a new machine through a process called
\concept{porting} away from the PDP machines it had run on previously.
This was aided by the fact \unix\ was written in C since much of the
code could simply be recompiled and didn't have to be rewritten.

In the late 1970's, AT\&T was forbidden from competing in the
computing industry, so it licensed \unix\ to various colleges and
universities very cheaply.  It was slow to catch on outside of
academic institutions but was eventually popular with businesses as
well. The \unix\ of today is different from the \unix\ of 1970.  It
has two major variations: System V, from Unix System Laboratories
(USL)\index{Unix System Laboratories}, a subsiderary of
Novell\footnote{It was recently sold to Novell.  Previously, USL was
  owned by AT\&T.\index{AT\&T}}, and the Berkeley Software
Distribution (BSD).  
\index{Novell}\index{BSD}\index{University of California, Berkeley} 
The USL version is now up to its forth release, or SVR4\footnote{A
  cryptic way of saying ``system five, release four''.}, while BSD's
latest version is 4.4.  However, there are many different versions of
\unix\ besides these two.  Most commercial versions of \unix\ derive
from one of the two groupings. The versions of \unix\ that
are actually used usually incorporate features from both variations.

Current commercial versions of \unix\ for Intel\index{Intel} PCs cost
between \$500 and \$2000.

\section{\linux\ History}

The primary author of \linux\ is Linus Torvalds\index{Torvalds, Linus}.  
Since his original versions, it has been improved by countless numbers
of people around the world.  It is a clone, written entirely from
scratch, of the Unix operating system.  Neither USL, nor the
University of California, Berkeley, were\index{Unix System
  Laboratories}\index{University of California, Berkeley} involved in
writing \linux.  One of the more interesting facts about \linux\ is
that development occurs simulataneously around the world.  People from
Austrialia to Finland contributed to \linux and will hopefully
continue to do so.

\linux\ began with a project to explore the 386 chip. One of Linus's
earlier projects was a program that would switch between printing {\tt
AAAA} and {\tt BBBB}. This later evolved to \linux.

\linux\ has been copyrighted under the terms of the GNU General Public
License (GPL)\index{General Public License}.  This is a license
written by the
Free Software Foundation (FSF)\index{Free Software Foundation} that
is designed to prevent people from restricting the distribution of
software.  In brief, it says that although you can charge as much as
you'd like for a copy, you can't prevent the person you sold it to
from giving it away for free. It also means that the source
code\footnote{The \concept{source code} of a program is what the
  programmer reads and writes.  It is later translated into unreadable
  machine code that the computer interprets.} must also be available.
This is useful for programmers.  Anybody can modify \linux\ and even
distributed his/her modifications, provided that they keep the code
under the same copyright.

\linux\ supports most of popular \unix\ software, including the X
Window System\index{X Window System}.  The X Window System was created
at the Massachusetts Institute of 
Technology\index{Massachusetts Institute of Technology}.  It was
written to allow \unix\ systems to create graphical windows and easily
interact with each other.  Today, the X Window System is used on every
version of \unix\ available.

In addition to the two variations of \unix, System V and BSD, there is
also a set of standardization documents published by the
IEEE\index{IEEE} entitled \concept{POSIX}.  \linux\ is first and
foremost compliant with the POSIX-1 and POSIX-2
documents.\index{POSIX}  Its look and feel is much like BSD in some
places, and somewhat like System V in others.  It is a blend (and to
most people, a good one) of all three standards.

Many of the utilities included with \linux\ distributions are from the
Free Software Foundation\index{Free Software Foundation} and are part
of GNU Project\index{GNU Project}.  The GNU Project is an effort to
write a portable, advanced operating system that will look a lot like
\unix.  ``Portable'' means that it will run on a variety of machines,
not just Intel\index{Intel} PCs, Macintoshes, or whatever.  The GNU
Project's operating system is called the Hurd\index{GNU Hurd}.  The
main difference between \linux\ and GNU Hurd is not in the user
interface but in the programmer's interface---the Hurd is a modern
operating system while \linux\ borrows more from the original \unix\
design.

The above history of \linux\ is deficient in mentioning anybody
\emph{besides} Linux Torvalds.  For instance, H.~J.~Lu\index{Lu,
  H.~J.} has maintained {\tt gcc} and the \linux\ C Library (two items
needed for all the programs on \linux) since very early in \linux's
life.  You can find a list of people who deserve to be recognized on
every \linux\ system in the file {\tt /usr/src/linux/CREDITS}.

\subsection{\linux\ Now}

The first number in \linux's version number indicates truly huge
revisions.  These change very slowly and as of this writing (February,
1996) only version ``1'' is available.  The second number indicates
less major revisions.  Even second numbers signify more stable,
dependable versions of \linux while odd numbers are developing
versions that are more prone to bugs.  The final version number is the
minor release number---every time a new version is released that may
just fix small problems or add minor features, that number is
increased by one.  As of February, 1996, the latest stable version is
1.2.11 and the latest development version is 1.3.61.

\linux\ is a large system and unfortunately contains bugs which are
found and then fixed.  Although some people still experience bugs
regularly, it is normally because of non-standard or faulty hardware;
bugs that effect everyone are now few and far between.

Of course, those are just the kernel bugs.  Bugs can be present in
almost every facet of the system, and inexperienced users have trouble
seperating different programs from each other. For instance, a problem
might arise that all the characters are some type of gibberish---is it
a bug or a ``feature''?  Surprisingly, this is a feature---the
gibberish is caused by certain control sequences that somehow
appeared.  Hopefully, this book will help you to tell the different
situations apart.

\subsection{A Few Questions and Answers}

% I'd like some trivial, and funny, questions and answers to put in
% here.  This isn't meant to be a serious section. Also, this may in
% the future be taken out of question/answer format. I personally
% think that's appropriate for a FAQ, but not for formal
% documentation. Perhaps as places become obvious, these paragraphs
% will move.

Before we embark on our long voyage, let's get the ultra-important out
of the way.

{\bf Question:} Just how do you pronounce \linux?\index{pronunciation}

{\bf Answer:} According to Linus\index{Torvalds, Linus}, it should be
pronounced with a short {\em ih\/} sound, like prInt, mInImal, etc.
\linux\ should rhyme with Minix, another Unix clone.  It should {\em
not\/} be pronounced like (American pronounciation of) the
``Peanuts''\index{Peanuts} character, Linus, but rather {\em
LIH-nucks}. And the {\em u} is sharp as in rule, not soft as in
ducks.  \linux\ should almost rhyme with ``cynics''.

{\bf Question:} Why work on \linux?

{\bf Answer:} Why not?  \linux\ is generally cheaper (or at least no
more expensive) than other operating systems and is frequently less
problematic than many commercial systems.  It might not be the best
system for your particular applications, but for someone who is
interested in using \unix\ applications available on \linux, it is a
high-performance system.

\subsection{Commercial Software in \linux}

There is a lot of commercial software available for \linux.  Starting
with Motif\index{Motif}, a user interface for the X Window
System\index{X Window System} that vaguely resembles Microsoft
Windows\index{Microsoft Windows}, \linux\ has been gaining more and
more commercial software.  These days you can buy anything from Word
Perfect (a popular word processor) to Maple, a complex symbolic
manipulation package, for \linux.

For any readers interested in the legalities of \linux, this is
allowed by the \linux\ license.  While the GNU General Public License
\index{General Public License}\index{Library General Public License}
(reproduced in Appendix~\ref{GNU-GPL}) covers the \linux\ kernel and
would seemingly bar commercial software, the
GNU Library General Public License (reproduced in
Appendix~\ref{GNU-LGPL}) covers most of the computer code applications
depend on.
\glossary{GPL}\glossary{GNU General Public License}
\glossary{LGPL}\glossary{GNU Library General Public License}
This allows commercial software providers to sell their applications
and withhold the source code.

Please note that those two documents are copyright notices, and not
licenses to use. They do {\em not\/} regulate how you may use the
software, merely under what circumstances you can copy it and any
derivative works.  To the Free Software Foundation, this is an
important distinction: \linux\ doesn't involve any ``shrink-wrap''
licenses but is merely protected by the same law that keeps you from
photocopying a book.

% Local Variables: 
% mode: latex
% TeX-master: "guide"
% End: 
